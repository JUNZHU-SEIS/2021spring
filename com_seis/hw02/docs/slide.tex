\documentclass{article}
\title{RAY BENDIHNG \& TRAVEL TIME}
\author{Jun Zhu, SA20007073}
\usepackage{graphicx}
\graphicspath{{../image/}}
\usepackage{listings}
\usepackage{color}
\definecolor{dkgreen}{rgb}{0,0.6,0}
\definecolor{gray}{rgb}{0.5,0.5,0.5}
\definecolor{mauve}{rgb}{0.58,0,0.82}
\lstset{frame=tb,
  language=Python,
  aboveskip=3mm,
  belowskip=3mm,
  showstringspaces=false,
  columns=flexible,
  basicstyle={\small\ttfamily},
  numbers=left,%设置行号位置none不显示行号
  %numberstyle=\tiny\courier, %设置行号大小
  numberstyle=\tiny\color{gray},
  keywordstyle=\color{blue},
  commentstyle=\color{dkgreen},
  stringstyle=\color{mauve},
  breaklines=true,
  breakatwhitespace=true,
  escapeinside=``,%逃逸字符(1左面的键),用于显示中文例如在代码中`中文...`
  tabsize=4,
  extendedchars=false %解决代码跨页时,章节标题,页眉等汉字不显示的问题
}
\begin{document}
\maketitle
\begin{lstlisting}
clear all
close all
global ray npts 
%po: position of source
po.x=0; po.z=12.81;
%ps: positio of receiver
ps.x=18.14; ps.z=-1.07;
%grid cofigurations
nx=100;nz=100;dx=1;dz=1;
%constant velocity model
%vz: velocity along the z axis
%vx: velocity along the x axis
%v0: 
vz=0; vx=1; v0=3;
[Velo, Vx, Vz]=v_field(nx,nz,vx,vz,v0,dx,dz);
%add random velocity field
[vxp, vzp]=add_random(nz,nx);
Vx = Vx + vxp; Vz = Vz + vzp;
max_npts=1000;
initial(po,ps,max_npts,v0,vx,vz,dx,dz,Velo,Vx,Vz);
%100: maximum iteration number
%1.e-8: termination threshold of updating travel time
getRay(100, 1.e-8);
t1 = traveltime(npts,ray)%  
\end{lstlisting}
\begin{lstlisting}
function getRay(maxiter, thred_t)
%max iteration number
%threshold of updated time difference
global ray npts
while(1)
    niter=0;
    while(1)
        niter=niter+1;
        if(niter>=maxiter)
            break;
        end
        t0 = traveltime(npts,ray);
        perturbRay();
  
        t1 = traveltime(npts,ray);
        if(abs(t0-t1)<thred_t )
            break;
        end
    end
    if(2*npts-1>=maxiter)
        break;
    end
    doublePath();
    t2 = traveltime(npts,ray);
    if(abs(t2-t1)<thred_t )
        break;
    end
    plotpath(ray);hold on
end            
\end{lstlisting}
\begin{lstlisting}
#include <stdio.h>
#include <stdlib.h>
#include "layer.h"
#include "numc.h"

#define N 100000
#define LN 130
int main(int argc,char *argv[])
{
	//read ak135 
	float d, v, dep[136], h[135], vp[136];
	int i=0;
	FILE *fp;
	if ((fp=fopen("../../data/AK135CSV/ak135","r")) == NULL)
	{
		printf("\nerror on open the file");
		getchar();
		exit(0);
	}
	while(!feof(fp))	
	{
		fscanf(fp,"%f %f",&d,&v);
		dep[i]=d;
		vp[i]=v;
		i++;
//		printf("%f %f\n",d,v);
	}
	for (i=0;i<135;i++)
	{
		h[i]=dep[i+1]-dep[i];
//		printf("%f\n",h[i]);
	}

	int irtr,j;
	float dx=0,dt=0,p[N],x[N]={0.0},t[N]={0.0},tau[N]={0.0};
	
	linspace(0.05434,0.08979,N,p);

	for(i=0;i<N;i++)
	{
		for(j=0;j<LN;j++)
		{
			irtr = layerxt(p[i],h[j],1.0/vp[j],1.0/vp[j+1],&dx,&dt);
			x[i] += dx;
			t[i] += dt;
		}
		tau[i] = t[i]-p[i]*x[i];
	}
	printf("param delta tt tau\n");
	fp=fopen("../../output/tttable.txt","w");
	fprintf(fp, "param delta tt tau\n");
	for(i=0;i<N;i++)
	{
		printf("%f	%f	%f	%f\n",p[i],x[i]/111,t[i],tau[i]);
		fprintf(fp, "%f	%f	%f	%f\n",p[i],x[i]/111,t[i],tau[i]);
	}
	fclose(fp);
	return 0;
}
\end{lstlisting}
\end{document}
