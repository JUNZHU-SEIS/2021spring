\documentclass{article}
\title{\textbf{hw02}}
\author{\textbf{Jun ZHU}}
\usepackage{graphicx}
\usepackage{epstopdf}
\graphicspath{{../image/}}
\usepackage{float}

\usepackage{xcolor}
\usepackage{listings}

\definecolor{mGreen}{rgb}{0,0.6,0}
\definecolor{mGray}{rgb}{0.5,0.5,0.5}
\definecolor{mPurple}{rgb}{0.58,0,0.82}
\definecolor{backgroundColour}{rgb}{0.95,0.95,0.92}

\lstdefinestyle{CStyle}{
    backgroundcolor=\color{backgroundColour},   
    commentstyle=\color{mGreen},
    keywordstyle=\color{magenta},
    numberstyle=\tiny\color{mGray},
    stringstyle=\color{mPurple},
    basicstyle=\footnotesize,
    breakatwhitespace=false,         
    breaklines=true,                 
    captionpos=b,                    
    keepspaces=true,                 
    numbers=left,                    
    numbersep=5pt,                  
    showspaces=false,                
    showstringspaces=false,
    showtabs=false,                  
    tabsize=2,
    language=C
}

\begin{document}
\maketitle
\section{TRAVEL TIME TABLE (FLATTENING MODEL)}
\subsection{FORMULA}

[X, T]=Ray1D(p,$\alpha$)\newline
give p\newline
$$X=2\sum_{j=1}^{N}Th_j\frac{p}{\eta_j} $$
$$T=2\sum_{j=1}^{N}\frac{Th_j}{\eta_j}\frac{1}{\alpha_j^2} $$
end
\subsection{SUBROUTINE}
\begin{lstlisting}
#include <math.h>

int layerxt(float p,float h,float utop,float ubot,float *dx,float *dt)
{
	int irtr;
	float eta1,eta2,tau1,tau2,dtau,x1,x2,u1,u2,v1,v2,b;
	if(p > utop)
	{
		*dx=0.0;
		*dt=0.0;
		irtr=0;
		return irtr;
	}
	else if(h == 0.0)
	{
		*dx=0.0;
		*dt=0.0;
		irtr=-1;
		return irtr;
	}

	u1=utop;
	u2=ubot;
	v1=1./u1;
	v2=1./u2;
	b=(v2-v1)/h;

	eta1=sqrt(u1*u1-p*p);

	if(b ==0)
	{
		*dx=h*p/eta1;
		*dt=h*u1*u1/eta1;
		irtr=1;
		return irtr;
	}

	x1=eta1/(u1*b*p);
	tau1=(log((u1+eta1)/p)-eta1/u1)/b;

	if(p > ubot)
	{
		*dx=x1;
		dtau=tau1;
		*dt=dtau+p*(*dx);
		irtr=2;
		return irtr;
	}

	irtr=1;

	eta2=sqrt(u2*u2-p*p);
	x2=eta2/(u2*b*p);
	tau2=(log((u2+eta2)/p)-eta2/u2)/b;

	*dx=x1-x2;
	dtau=tau1-tau2;

	*dt=dtau+p*(*dx);

	return irtr;
}
\end{lstlisting}
\subsection{MAIN FUNCTION}
\begin{lstlisting}
#include <stdio.h>
#include <stdlib.h>
#include "layer.h"
#include "numc.h"

#define N 100000
#define LN 130
int main(int argc,char *argv[])
{
	//read ak135 
	float d, v, dep[136], h[135], vp[136];
	int i=0;
	FILE *fp;
	if ((fp=fopen("../../data/AK135CSV/ak135","r")) == NULL)
	{
		printf("\nerror on open the file");
		getchar();
		exit(0);
	}
	while(!feof(fp))	
	{
		fscanf(fp,"%f %f",&d,&v);
		dep[i]=d;
		vp[i]=v;
		i++;
//		printf("%f %f\n",d,v);
	}
	for (i=0;i<135;i++)
	{
		h[i]=dep[i+1]-dep[i];
//		printf("%f\n",h[i]);
	}

	int irtr,j;
	float dx=0,dt=0,p[N],x[N]={0.0},t[N]={0.0},tau[N]={0.0};
	
	linspace(0.05434,0.08979,N,p);

	for(i=0;i<N;i++)
	{
		for(j=0;j<LN;j++)
		{
			irtr = layerxt(p[i],h[j],1.0/vp[j],1.0/vp[j+1],&dx,&dt);
			x[i] += dx;
			t[i] += dt;
		}
		tau[i] = t[i]-p[i]*x[i];
	}
	printf("param delta tt tau\n");
	fp=fopen("../../output/tttable.txt","w");
	fprintf(fp, "param delta tt tau\n");
	for(i=0;i<N;i++)
	{
		printf("%f	%f	%f	%f\n",p[i],x[i]/111,t[i],tau[i]);
		fprintf(fp, "%f	%f	%f	%f\n",p[i],x[i]/111,t[i],tau[i]);
	}
	fclose(fp);
	return 0;
}
\end{lstlisting}
\section{RAY TRACING}
\subsection{BENDING}
\begin{figure}[H]
  \centering
  \includegraphics{raypath0.png}
  \caption{Ray tracing by bendding method: NO RANDOM VELOCITY FIELD}
\end{figure}
\begin{figure}[H]
  \centering
  \includegraphics{raypath0.1.png}
  \caption{Ray tracing by bendding method: 10\% RANDOM VELOCITY FIELD LOADED}
\end{figure}
\begin{figure}[H]
  \centering
  \includegraphics{raypath0.3.png}
  \caption{Ray tracing by bendding method: 30\% RANDOM VELOCITY FIELD LOADED}
\end{figure}
\begin{figure}[H]
  \centering
  \includegraphics{raypath0.5.png}
  \caption{Ray tracing by bendding method: 50\% RANDOM VELOCITY FIELD LOADED}
\end{figure}
\begin{figure}[H]
  \centering
  \includegraphics{raypath0.8.png}
  \caption{Ray tracing by bendding method: 80\% RANDOM VELOCITY FIELD LOADED}
\end{figure}
\begin{figure}[H]
  \centering
  \includegraphics{raypath1.png}
  \caption{Ray tracing by bendding method: 100\% RANDOM VELOCITY FIELD LOADED}
\end{figure}
\subsection{SUBROUTINE}
\begin{lstlisting}
function getRay(maxiter, thred_t)
%max iteration number
%threshold of time
global ray npts
while(1)
    niter=0;
    while(1)
        niter=niter+1;
        if(niter>=maxiter)
            break;
        end
        t0 = traveltime(npts,ray);
        perturbRay();

        t1 = traveltime(npts,ray);
        if(abs(t0-t1)<thred_t )
            break;
        end
    end
    if(2*npts-1>=maxiter)
        break;
    end
    doublePath();
    t2 = traveltime(npts,ray);
    if(abs(t2-t1)<thred_t )
        break;
    end
    plotpath(ray);hold on
end
\end{lstlisting}
\end{document}
