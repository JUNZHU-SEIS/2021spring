\documentclass{article}
\usepackage{palatino}
\usepackage{tikz}
\usetikzlibrary{shapes.geometric, arrows}
\begin{document}
\thispagestyle{empty}
% 流程图定义基本形状
\tikzstyle{startstop} = [rectangle, rounded corners, minimum width = 2cm, minimum height=1cm,text centered, draw = black]
\tikzstyle{io} = [trapezium, trapezium left angle=70, trapezium right angle=110, minimum width=2cm, minimum height=1cm, text centered, draw=black]
\tikzstyle{process} = [rectangle, minimum width=3cm, minimum height=1cm, text centered, draw=black]
\tikzstyle{check} = [diamond, aspect = 3, text centered, draw=black]
% 箭头形式
\tikzstyle{arrow} = [->,>=stealth]
\begin{tikzpicture}[node distance=5mm]
%定义流程图具体形状
\node[startstop](start){Start subroutine: source inversion \textbf{for given depth}};
\node[io, below of = start, yshift = -1cm](in1){Input search range of $\theta, \lambda, \delta$ \& depth $h$};
\node[process, below of = in1, yshift = -1cm](pro1){Synthetics \textbf{for given depth $h$}};
\node[process, below of = pro1, yshift = -1cm](pro2){Segmentation \& Crosscorrelation \& Alignment};
\node[process, below of = pro2, yshift = -1cm](pro3){Calculate $M_0$ for \textbf{each common time segment} and $\bar {M_0}$ for each station};
\node[process, below of = pro3, yshift = -1cm](pro4){Minimize the average misfit of all stations};
\node[io, below of = pro4, yshift = -1cm](out1){Output optimal solution $\theta_0, \lambda_0, \delta_0, \bar {\bar {M_0}}$};
\node[startstop, below of = out1, yshift = -1cm](stop){Stop};
%\coordinate (point1) at (-5cm, -6cm);
%%连接具体形状
\draw [arrow] (start) -- (in1);
\draw [arrow] (in1) -- (pro1);
\draw [arrow] (pro1) -- (pro2);
\draw [arrow] (pro2) -- (pro3);
\draw [arrow] (pro3) -- (pro4);
\draw [arrow] (pro4) -- (out1);
\draw [arrow] (out1) -- (stop);
\end{tikzpicture}
\end{document}
