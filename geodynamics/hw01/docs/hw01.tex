\documentclass{article}
\title{\textbf{hw01}}
\author{\textbf{Jun Zhu, SA20007073}}
\usepackage{graphicx}
\usepackage{float}
\graphicspath{{../image/}}
\begin{document}
\maketitle
\begin{figure}[H]
\centering
\includegraphics[scale=1.2]{global_ETOP.png}
\caption{global topography}
\end{figure}
\begin{figure}[H]
\centering
\includegraphics{regional_ETOP.png}
\caption{regional topography}
\end{figure}
\begin{figure}[H]
\centering
\includegraphics{global_quake_distribution.png}
\caption{global earthquake distribution}
\end{figure}
\begin{figure}[H] 
\centering
\includegraphics{regional_quake_distribution.png}
\caption{regional earthquake distribution}
\end{figure}
\begin{figure}[H]
\centering
\includegraphics[scale=0.7]{Tonga.png}
\caption{map view of earthquake distribution in Tonga, where the red line denotes the traverse line of the Benioff zone}
\end{figure}
\begin{figure}[H]
\centering
\includegraphics{subduction.png}
\caption{vertical profile of earthquake distribution in Tonga}
\end{figure}
\end{document}
