\documentclass{article}
\title{\textbf{hw03 for Geodynamics}}
\author{\textbf{Jun ZHU, SA20007073}}
\usepackage{amsmath}
\usepackage{listings}
\usepackage{color}
\definecolor{dkgreen}{rgb}{0,0.6,0}
\definecolor{gray}{rgb}{0.5,0.5,0.5}
\definecolor{mauve}{rgb}{0.58,0,0.82}
\lstset{frame=tb,
  language=Python,
  aboveskip=3mm,
  belowskip=3mm,
  showstringspaces=false,
  columns=flexible,
  basicstyle={\small\ttfamily},
  numbers=none,
  numberstyle=\tiny\color{gray},
  keywordstyle=\color{blue},
  commentstyle=\color{dkgreen},
  stringstyle=\color{mauve},
  breaklines=true,
  breakatwhitespace=true,
  tabsize=3
}
\usepackage{graphicx}
\graphicspath{{../image/}}
\usepackage{float}
\begin{document}
\maketitle
\section{theorem}
\subsection{equat
ions relating steady-state concentrated loading and crustal response}
\begin{equation}\label{Brotchie}
D\nabla^4\omega+(ET/R^2)\omega+\gamma\omega=q
\end{equation}
in which:\newline
$\omega$: the radial displacement\newline
$q$: normal loading of intensity\newline
$D$: flexural stiffness of the shell cross section\newline
\begin{equation}
D=ET^3/12(1-\nu^2)
\end{equation}
$T$: thickness of the shell\newline
$E$: Young's modulus\newline
$\nu$: Possion's ratio\newline
\textbf{$R$: radius of its middle surface\newline}
$\gamma$: density of the liquid\newline
$\nu$: Possion's ratio\newline
$\nabla^4$: biharmonic operator\newline

Rewrite the equation \ref{Brotchie}:

\begin{equation}\label{rewritten}
\nabla^4\omega+(1/l^4)\omega=q/D
\end{equation}

in which:\newline
\textbf{$l$: is the radius of relative stiffness}

\begin{equation}
l^4=D/[(ET/R^2)+\gamma]
\end{equation}

The homogeneous (basic) solution:\newline
\begin{subequations}\label{solution}
\begin{align*}
&\omega_0=C_1ber(x)+C_2bei(x)+C_3ker(x)+C_4kei(x)\\ 
&\omega_P=ql^4/D\\
&\omega=\omega_0+\omega_P
\end{align*}
\end{subequations}
in which:\newline
$x$: $r/l$

The solution for a concentrated load of magnitude P:\newline

\begin{equation}\label{volcano}
\omega=(Pl^2/2\pi D)kei(x)
\end{equation}

in which:\newline
$P$: total loading due to the volcanic cone\newline

The solution for uniform loading:\newline
\begin{equation}
  \omega_i=\frac{\gamma_{ice}h}{\gamma^{'}}(aker^{'}(a)ber(x)-akei^{'}(a)bei(x)+1)
\end{equation}
\begin{equation}
  \omega_0=\frac{\gamma_{ice}h}{\gamma^{'}}(aber^{'}(a)ker(x)-abei^{'}(a)kei(x))
\end{equation}


in which:\newline
$a=A/l$\newline

\begin{figure}[H]
\centering
\includegraphics[scale=0.5]{schematic.png}
\caption{schematic for r, l}
\end{figure}


\section{code setup}
\begin{lstlisting}
import numpy as np
class Config():
	def __init__(self):
		self.rate = 300 * 1e3
		self.duration = 2
		self.height = 2 * 1e3
		self.up2bot = 0.8
		#unit conversion Gpa2pa
		self.E = 70 * 1e9
		self.possion = 0.25
		#unit conversion
		self.rhoBasalts = 2700 * 9.8065
		self.gamma = 2900 * 9.8065
		#thickness
		self.T = 100 * 1e3
		self.r = np.arange(0, 100, 10) * 1e3
		self.delta = 0.01

		self.R = 6371 * 1e3 - self.T
		self.pi = 3.141592654
		self.t0 = 0
		self.te = 2

		self.GAMMA = self.gamma + self.E * self.T / self.R**2
		self.GammaRatio = self.gamma / self.GAMMA
		self.D = self.E*self.T**3/12/(1 - self.possion**2)
		self.l = (self.D / ((self.E*self.T/self.R**2) + self.gamma))**0.25
		self.x = np.array(self.r) / self.l

		\end{lstlisting}
\section{figure}

\begin{figure}[H]
\centering
\includegraphics[scale=0.8]{100T100km.png}
\caption{T=100km}
\end{figure}

\begin{figure}[H]
\centering
\includegraphics[scale=0.8]{100T50km.png}
\caption{T=50km}
\end{figure}

\begin{figure}[H]
\centering
\includegraphics[scale=0.8]{500T100km.png}
\caption{T=100km}
\end{figure}

\begin{figure}[H]
\centering
\includegraphics[scale=0.8]{500T50km.png}
\caption{T=50km}

\end{figure}


\end{document}
