\documentclass{article}
\title{\textbf{HW03}}
\author{\textbf{Jun Zhu, SA20007073}}
\usepackage{graphicx}
\usepackage{float}
\graphicspath{{image/}}

\usepackage{indentfirst}

\usepackage{listings}
\usepackage{color}

\definecolor{dkgreen}{rgb}{0,0.6,0}
\definecolor{gray}{rgb}{0.5,0.5,0.5}
\definecolor{mauve}{rgb}{0.58,0,0.82}

\lstset{frame=tb,
  language=Python,
  aboveskip=3mm,
  belowskip=3mm,
  showstringspaces=false,
  columns=flexible,
  basicstyle={\small\ttfamily},
  numbers=none,
  numberstyle=\tiny\color{gray},
  keywordstyle=\color{blue},
  commentstyle=\color{dkgreen},
  stringstyle=\color{mauve},
  breaklines=true,
  breakatwhitespace=true,
  tabsize=3
}

\setlength{\parindent}{2em}
\begin{document}
\maketitle
\newpage{}
\section{}
$r_v$ is used for geophone, while $r_p$ is used for hydrophone.
Geophone records the velocity of the particle vibration.

The design of effective hydrophones must take into account the acoustic resistance of water, 
which is 3750 times that of air; 
hence the pressure exerted by a wave of the same intensity in air is increased by a factor of 3750 in water.
If any problem recording the pressure, maybe removing the acoustic response of different mediums is a feasible solution.
\section{Multiplicity Analysis}
\subsection{Some figures}
\begin{figure}[H]
  \centering
  \includegraphics{shot_p0shot_n10shot_i20offset20receiver_n12receiver_i10.png}
  \caption{shot position: 0, shot number: 10, shot interval: 20, offset: 20, receiver number: 20, receiver interval: 10}
  \label{ref}
\end{figure}
\begin{figure}[H]
  \centering
  \includegraphics{shot_p0shot_n10shot_i20offset20receiver_n12receiver_i1.png}
  \caption{except for receiver interval equals to 1, the others are the same with Figure 1}
\end{figure}
\begin{figure}[H]
  \centering
  \includegraphics{shot_p0shot_n10shot_i20offset20receiver_n12receiver_i20.png}
  \caption{except for receiver interval equals to 20, the others are the same with Figure 1}
\end{figure}
\subsection{Python code (Numpy and Matplotlib are required)}

\begin{lstlisting}
# Author: Jun Zhu
# Date:   MAR, 28 2021
# Email:  Jun__Zhu@outlook.com

import numpy as np
import matplotlib.pyplot as plt
from matplotlib.ticker import MaxNLocator

def inwindow(x=[4,4.1],y=[2,4]):
	"""check if the box is in the window
		notes:
			if True, return 1
			if False, return 0
	"""
	if (x[1]>y[0]) and (x[0]<y[1]):
		tick = 1
	else:
		tick =0
	return tick

def fold(
		shot_p=0,
		shot_n=2,
		shot_i=1,
		offset=1,
		receiver_n=2,
		receiver_i=1):
	"""calculate multiplicity
	"""
	first_CMPs = []
	CMP_lists = []
	CMPs = []
	shot_pn = shot_p
	for n in range(shot_n):
		first_CMP = shot_pn + 1/2*offset
		first_CMPs.append(first_CMP)
		CMP_n = first_CMP + np.arange(receiver_n)*receiver_i/2
		CMP_lists.append(CMP_n)
		CMPs += list(CMP_n)
		shot_pn +=  shot_i
	CMPs = list(set(CMPs))
	CMPs.sort()
	boxes = [[x, CMPs[i+1]] for i,x in enumerate(CMPs[:-1])]
	boxes = [[shot_p, first_CMPs[0]]] + boxes
	windows = [[x[0],x[-1]] for x in CMP_lists]
	multiplicity = [sum([inwindow(box,window) for window in windows]) for box in boxes]
	n = multiplicity
	lastbox_width = boxes[-1][-1]-boxes[-1][0]
	bins = [x[0] for x in boxes]
	width = np.append(np.diff(bins), lastbox_width)
	ax = plt.figure().gca()
	ax.bar(bins, n, width=width, align='edge', edgecolor='k', color='r')
	ax.axvline(x=shot_p, ls='--', c='gray')
	ax.set_xlabel('distance')
	ax.set_ylabel('fold')
	parameters = "shot_p"+str(shot_p)+"shot_n"+str(shot_n)+"shot_i"+str(shot_i)+"offset"+str(offset)+"receiver_n"+str(receiver_n)+"receiver_i"+str(receiver_i)
	ax.set_title(parameters)
	ax.yaxis.set_major_locator(MaxNLocator(integer=True))
	ax.grid(linestyle='-.', axis='y')
	fname = "../image/"+parameters+".png"
	plt.savefig(fname, dpi=500)
	#plt.show()
	pass

if __name__ == "__main__":
	candidate_receiver_i = np.arange(20) + 1
	for receiver_i in candidate_receiver_i:
		fold(
			shot_p=0,
			shot_n=10,
			shot_i=20,
			offset=20,
			receiver_n=12,
			receiver_i=receiver_i)
	pass

	\end{lstlisting}
\end{document}
