\documentclass{article}
\title{\textbf{HW03: model test}}
\author{\textbf{Jun ZHU, SA20007073}}
%\date{March 25, 2021}
\usepackage{amsmath}
\usepackage{hyperref}
\usepackage{graphicx}
\usepackage{float}
\graphicspath{{../image/}}
\usepackage{indentfirst}
\setlength{\parindent}{2em}
\usepackage{listings}
\usepackage{color}
\definecolor{mygreen}{RGB}{28,172,0}
\definecolor{mylilas}{RGB}{170,55,241}

\lstset{language=Matlab,
    breaklines=true,
    morekeywords={matlab2tikz},
    keywordstyle=\color{blue},
    morekeywords=[2]{1}, keywordstyle=[2]{\color{black}},
    identifierstyle=\color{black},
    stringstyle=\color{mylilas},
    commentstyle=\color{mygreen},
    showstringspaces=false,
    numbers=left,
    numberstyle={\tiny \color{black}},
    numbersep=9pt,
    emph=[1]{for,end,break},emphstyle=[1]\color{red},
    emph=[2]{word1,word2}, emphstyle=[2]{style},    
}

\begin{document}
\maketitle
\section{theoretical model}
\includegraphics[scale=0.2]{eq_velocity.png}\newline
\includegraphics[scale=0.2]{eq_raypath.png}\newline
\includegraphics[scale=0.2]{eq_turningtime.png}
\section{model setup}
$$v_0=1600\;m/s$$
$$K=0.5\;s^{-1}$$
$$\alpha _{0}=[0:\pi/200:\pi/2]$$
\section{ray path}
\subsection{tt table method vs. theoretical}
\begin{figure}[H]
  \centering
  \includegraphics[scale=0.4]{modeltestable.jpg}
  \caption{Black lines denote theoretical ray paths with take-off angle $\alpha _0$ ranges from 0 to $\pi/2$, the white line denotes the ray path calculated by travel time table steepest descent method.}
\end{figure}
\subsection{two-point ray-tracing method vs. theoretical}
\begin{figure}[H]
  \centering
  \includegraphics[scale=0.4]{modeltestangle.jpg}
  \caption{Black lines denote theoretical ray paths with take-off angle $\alpha _0$ ranges from 0 to $\pi/2$, the white line denotes the optimal ray path calculated by two-point ray-tracing method.}
\end{figure}
\section{Supplement}
\subsection{how to set the vertical linear velocity model \& calculate theoretical ray path}
\begin{lstlisting}
v0=1600; K=0.5; z=(1:N1)*h; beta=K/(v0+K*z(2));
vz=v0+K*z;
vel=vz'.*ones(1,N2);
deltangl=pi/200;
alpha0=[0:deltangl:pi/2];
x1=1./(beta*tan(alpha0))+h*10;
z1=-1/beta+h*2;
r1=1./(beta*sin(alpha0));
theta=[0:pi/2^10:2*pi];
u=x1+r1.*cos(theta');
v=z1+r1.*sin(theta');
\end{lstlisting}
\end{document}
